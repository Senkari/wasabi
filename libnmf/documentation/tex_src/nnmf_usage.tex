\documentclass[a4paper,10pt]{scrartcl}
\usepackage[utf8]{inputenc}
\usepackage{amsmath}
\usepackage{fancyhdr}
\pagestyle{fancy}
\usepackage{dsfont}
\usepackage{amsfonts}
\usepackage{amsthm}
\usepackage{graphicx}
\usepackage[small,nooneline,bf,hang]{caption2}
\usepackage{float}
\usepackage{hyperref}

%opening
\title{Usage of NNMF C-Library 1.02}



\begin{document}

\fancyhead{}
\rhead{Usage}
\lhead{NNMF C-Library 1.02}


\maketitle


	\section{Building the library}

	The library can be build via the provided makefile, which by default uses the ``gcc'' compiler.\newline\newline
	The target \emph{lib} builds the static library and target \emph{shared} builds the shared library.
	By default the makefile compiles without the compiler flag \emph{-fPIC} (position independent code) which is
	needed for generating the shared library.\newline\newline
	Optionally one could alter compiler flags in the makefile before building the library.

	\section{Linking the library}

	Linking the library requires linking of ARPACK, BLAS and LAPACK libraries as well.

	\section{Usage}

	This library can be used by either calling \emph{nmfDriver} or the individual computational routines for
	a non negative matrix factorization.\newline

	An example call for the first case can be found in the file ``example.c''.\newline

	An example call for the second case can be found in the file ``example\_withoutdriver.c''\newline

	

	
\end{document}
